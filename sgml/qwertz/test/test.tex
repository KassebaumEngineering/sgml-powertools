\documentstyle[qwertz,dina4,xlatin1,11pt]{report}
%   EPSF.TEX macro file:
%   Written by Tomas Rokicki of Radical Eye Software, 29 Mar 1989.
%   Revised by Don Knuth, 3 Jan 1990.
%   Revised by Tomas Rokicki to accept bounding boxes with no
%      space after the colon, 18 Jul 1990.
%
%   TeX macros to include an Encapsulated PostScript graphic.
%   Works by finding the bounding box comment,
%   calculating the correct scale values, and inserting a vbox
%   of the appropriate size at the current position in the TeX document.
%
%   To use with the center environment of LaTeX, preface the \epsffile
%   call with a \leavevmode.  (LaTeX should probably supply this itself
%   for the center environment.)
%
%   To use, simply say
%   \input epsf           % somewhere early on in your TeX file
%   \epsfbox{filename.ps} % where you want to insert a vbox for a figure
%
%   Alternatively, you can type
%
%   \epsfbox[0 0 30 50]{filename.ps} % to supply your own BB
%
%   which will not read in the file, and will instead use the bounding
%   box you specify.
%
%   The effect will be to typeset the figure as a TeX box, at the
%   point of your \epsfbox command. By default, the graphic will have its
%   `natural' width (namely the width of its bounding box, as described
%   in filename.ps). The TeX box will have depth zero.
%
%   You can enlarge or reduce the figure by saying
%     \epsfxsize=<dimen> \epsfbox{filename.ps}
%   (or
%     \epsfysize=<dimen> \epsfbox{filename.ps})
%   instead. Then the width of the TeX box will be \epsfxsize and its
%   height will be scaled proportionately (or the height will be
%   \epsfysize and its width will be scaled proportiontally).  The
%   width (and height) is restored to zero after each use.
%
%   A more general facility for sizing is available by defining the
%   \epsfsize macro.    Normally you can redefine this macro
%   to do almost anything.  The first parameter is the natural x size of
%   the PostScript graphic, the second parameter is the natural y size
%   of the PostScript graphic.  It must return the xsize to use, or 0 if
%   natural scaling is to be used.  Common uses include:
%
%      \epsfxsize  % just leave the old value alone
%      0pt         % use the natural sizes
%      #1          % use the natural sizes
%      \hsize      % scale to full width
%      0.5#1       % scale to 50% of natural size
%      \ifnum#1>\hsize\hsize\else#1\fi  % smaller of natural, hsize
%
%   If you want TeX to report the size of the figure (as a message
%   on your terminal when it processes each figure), say `\epsfverbosetrue'.
%
\newread\epsffilein    % file to \read
\newif\ifepsffileok    % continue looking for the bounding box?
\newif\ifepsfbbfound   % success?
\newif\ifepsfverbose   % report what you're making?
\newdimen\epsfxsize    % horizontal size after scaling
\newdimen\epsfysize    % vertical size after scaling
\newdimen\epsftsize    % horizontal size before scaling
\newdimen\epsfrsize    % vertical size before scaling
\newdimen\epsftmp      % register for arithmetic manipulation
\newdimen\pspoints     % conversion factor
%
\pspoints=1bp          % Adobe points are `big'
\epsfxsize=0pt         % Default value, means `use natural size'
\epsfysize=0pt         % ditto
%
\def\epsfbox#1{\global\def\epsfllx{72}\global\def\epsflly{72}%
   \global\def\epsfurx{540}\global\def\epsfury{720}%
   \def\lbracket{[}\def\testit{#1}\ifx\testit\lbracket
   \let\next=\epsfgetlitbb\else\let\next=\epsfnormal\fi\next{#1}}%
%
\def\epsfgetlitbb#1#2 #3 #4 #5]#6{\epsfgrab #2 #3 #4 #5 .\\%
   \epsfsetgraph{#6}}%
%
\def\epsfnormal#1{\epsfgetbb{#1}\epsfsetgraph{#1}}%
%
\def\epsfgetbb#1{%
%
%   The first thing we need to do is to open the
%   PostScript file, if possible.
%
\openin\epsffilein=#1
\ifeof\epsffilein\errmessage{I couldn't open #1, will ignore it}\else
%
%   Okay, we got it. Now we'll scan lines until we find one that doesn't
%   start with %. We're looking for the bounding box comment.
%
   {\epsffileoktrue \chardef\other=12
    \def\do##1{\catcode`##1=\other}\dospecials \catcode`\ =10
    \loop
       \read\epsffilein to \epsffileline
       \ifeof\epsffilein\epsffileokfalse\else
%
%   We check to see if the first character is a % sign;
%   if not, we stop reading (unless the line was entirely blank);
%   if so, we look further and stop only if the line begins with
%   `%%BoundingBox:'.
%
          \expandafter\epsfaux\epsffileline:. \\%
       \fi
   \ifepsffileok\repeat
   \ifepsfbbfound\else
    \ifepsfverbose\message{No bounding box comment in #1; using defaults}\fi\fi
   }\closein\epsffilein\fi}%
%
%   Now we have to calculate the scale and offset values to use.
%   First we compute the natural sizes.
%
\def\epsfsetgraph#1{%
   \epsfrsize=\epsfury\pspoints
   \advance\epsfrsize by-\epsflly\pspoints
   \epsftsize=\epsfurx\pspoints
   \advance\epsftsize by-\epsfllx\pspoints
%
%   If `epsfxsize' is 0, we default to the natural size of the picture.
%   Otherwise we scale the graph to be \epsfxsize wide.
%
   \epsfxsize\epsfsize\epsftsize\epsfrsize
   \ifnum\epsfxsize=0 \ifnum\epsfysize=0
      \epsfxsize=\epsftsize \epsfysize=\epsfrsize
%
%   We have a sticky problem here:  TeX doesn't do floating point arithmetic!
%   Our goal is to compute y = rx/t. The following loop does this reasonably
%   fast, with an error of at most about 16 sp (about 1/4000 pt).
% 
     \else\epsftmp=\epsftsize \divide\epsftmp\epsfrsize
       \epsfxsize=\epsfysize \multiply\epsfxsize\epsftmp
       \multiply\epsftmp\epsfrsize \advance\epsftsize-\epsftmp
       \epsftmp=\epsfysize
       \loop \advance\epsftsize\epsftsize \divide\epsftmp 2
       \ifnum\epsftmp>0
          \ifnum\epsftsize<\epsfrsize\else
             \advance\epsftsize-\epsfrsize \advance\epsfxsize\epsftmp \fi
       \repeat
     \fi
   \else\epsftmp=\epsfrsize \divide\epsftmp\epsftsize
     \epsfysize=\epsfxsize \multiply\epsfysize\epsftmp   
     \multiply\epsftmp\epsftsize \advance\epsfrsize-\epsftmp
     \epsftmp=\epsfxsize
     \loop \advance\epsfrsize\epsfrsize \divide\epsftmp 2
     \ifnum\epsftmp>0
        \ifnum\epsfrsize<\epsftsize\else
           \advance\epsfrsize-\epsftsize \advance\epsfysize\epsftmp \fi
     \repeat     
   \fi
%
%  Finally, we make the vbox and stick in a \special that dvips can parse.
%
   \ifepsfverbose\message{#1: width=\the\epsfxsize, height=\the\epsfysize}\fi
   \epsftmp=10\epsfxsize \divide\epsftmp\pspoints
   \vbox to\epsfysize{\vfil\hbox to\epsfxsize{%
      \special{PSfile=#1 llx=\epsfllx\space lly=\epsflly\space
          urx=\epsfurx\space ury=\epsfury\space rwi=\number\epsftmp}%
      \hfil}}%
\epsfxsize=0pt\epsfysize=0pt}%

%
%   We still need to define the tricky \epsfaux macro. This requires
%   a couple of magic constants for comparison purposes.
%
{\catcode`\%=12 \global\let\epsfpercent=%\global\def\epsfbblit{%BoundingBox}}%
%
%   So we're ready to check for `%BoundingBox:' and to grab the
%   values if they are found.
%
\long\def\epsfaux#1#2:#3\\{\ifx#1\epsfpercent
   \def\testit{#2}\ifx\testit\epsfbblit
      \epsfgrab #3 . . . \\%
      \epsffileokfalse
      \global\epsfbbfoundtrue
   \fi\else\ifx#1\par\else\epsffileokfalse\fi\fi}%
%
%   Here we grab the values and stuff them in the appropriate definitions.
%
\def\epsfgrab #1 #2 #3 #4 #5\\{%
   \global\def\epsfllx{#1}\ifx\epsfllx\empty
      \epsfgrab #2 #3 #4 #5 .\\\else
   \global\def\epsflly{#2}%
   \global\def\epsfurx{#3}\global\def\epsfury{#4}\fi}%
%
%   We default the epsfsize macro.
%
\def\epsfsize#1#2{\epsfxsize}
%
%   Finally, another definition for compatibility with older macros.
%
\let\epsffile=\epsfbox

\pagestyle{headings}
\title{The {\tt qwertz} SGML Document Types
\\
{\large (Version 1.1 Reference Manual)}
}
\author{Tom Gordon \\
\\
The {\tt qwertz} Project \\ Institute for Applied Information Technology (F3) \\ \\ German National Research Center \\ for Computer Science (GMD)}


\begin{document}
\maketitle

\chapter{Why Not Just Use LaTeX?}

The {\tt qwertz} document types are a set of Standard Generalized
Markup Language (SGML) document type definitions (DTDs) for articles,
reports, books, letters, notes, slides (or overhead transparencies),
bibliographies, and manual pages.  Except for manual pages, the
document types have been heavily influenced by the LaTeX document
types of the same names \cite{Lamport86}, so LaTeX users should
feel right at home.  Indeed, we presently translate most {\tt qwertz}
documents into LaTeX for printing and the LaTeX produced is quite
readable by anyone familiar with LaTeX.

{\ldots}




\chapter{The {\tt qwertz} Document Type Definition}

\markboth{The {\tt qwertz} Document Types}{The {\tt qwertz} DTD}
{\ldots}

We will be making use of several {\em parameter entities\/} in this
DTD:

\par
\addvspace{\medskipamount}
\nopagebreak\hrule
\begin{verbatim}
<!entity % emph 
        " em | it | bf | sf | sl | tt " >

<!entity % inline 
        " f | x | %emph; | sq | label | ref | 
          pageref | cite | ncite " >
\end{verbatim} 
\nopagebreak\hrule 
\addvspace{\medskipamount}


{\ldots}




\section{General Purpose Entities and Elements}

\markboth{The {\tt qwertz} Document Types}{General Purpose Entities}
{\ldots}

When may it be necessary to use of an entity reference to produce
some character?  There are three cases to watch out for:

\begin{description}
\item[SGML Concrete Syntax Delimiters.] \mbox{}

Although the SGML standard allows alternative concrete syntaxes to
be defined, we use the so-called {\em reference concrete syntax\/} in
the {\tt qwertz} document types.  {\ldots}



\item[SGML Short Reference Delimiters.] \mbox{}

In SGML document types {\em short reference maps\/} may be defined
which allow single characters to be interpreted as arbitrarily complex
sequences of characters, including SGML tags and entity references. {\ldots}

\begin{verbatim}
" # % ' ( ) * + , - : ; = @ [ ] ^ _ { | } ~
\end{verbatim}


For each of these characters, there is an SGML entity which may be
used to generate the ASCII character in the printed document, listed
in table \ref{GPC}. {\em Usually, it will not be necessary to use these
entities; the character can simply be typed and will be interpreted
literally.\/} However, {\ldots}



\item[TeX Special Characters.] \mbox{}

Ideally, it should be possible to hide the conventions of the
underlying formatting system completely.  In fact, SGML parsers which
implement the full ISO standard have a feature which makes this
possible. {\ldots}



\end{description}





\subsection{Spacing, Dashes and Ellipsis}

\markboth{The {\tt qwertz} Document Types}{Spacing, Dashes and Ellipsis}
{\ldots}

There are also three different kinds of dashes: {\tt hyphen}
which was already mentioned above, is to be used for intra-word dashes, as
in the word ``intra-word''.\footnote{However, the {\tt hyphen} entity
was not actually necessary here, as the - character was not being used
in this context as a short reference.}

{\ldots}




\subsection{Foreign Characters}

\markboth{The {\tt qwertz} Document Types}{Foreign Characters}
There are a large set of entities for other Western European
languages.  Altogether, there are entities for almost all of the
foreign language characters in ISO 8859, the Latin 1 character set for
Western European languages.\footnote{Only the four
Icelandic characters are missing.} {\ldots}

\begin{table}[tbp]
\begin{center}
\begin{tabular}{ll|ll|ll|ll}
AElig &         {\AE} & Aacute &        � & Acirc       &       � & Ae &  �     \\ 
Ntilde &        \~{N} & Oacute &        � & Ocirc &     � & Oe &  � \\ 
Ue &  � & Ugrave &      � & Uuml &      � & Yacute &    � \\ 
aacute &        � & acirc &     � & ae &  � & aelig &   {\ae} \\ 
oe &  � & ograve &      � & oslash &    {\o} & otilde &         \^{o} \\ 
sz &  � & szlig &       � & thinsp &  & tilde &  \verb+~+ \\ 
times &  \mch{\times} & uacute &        � & ucirc &     � & ue &  � \\ 
\end{tabular}
\end{center}
\caption{\label{GPC} (Some) General Purpose Characters}
\end{table}


{\ldots}




\subsection{Sentences, Paragraphs, Footnotes and Emphasis}

\markboth{The {\tt qwertz} Document Types}{Sentences {\ldots} Emphasis}
{\ldots}

Sentences or phrases within paragraphs can be emphasized in a
number of ways.  The {\tt em} tag is used to choose the default form
of emphasis, which is usually {\em italic\/} type, but depends on the
style of the background text.  If the background text is formatted in
italics type, as it usually is in definitions, for example, than
emphasized text will be formatted using a plain, roman typeface.
However, various forms of emphasis can be explicitly chosen.  These
include: {\bf bold face} ({\tt bf}), {\it italics\/} ({\tt it}),
{\sf sans serif} ({\tt sf}), {\sl slanted} ({\tt sl}), and
{\tt typewriter} ({\tt tt}) styles.

{\ldots}

Long quotes are formatted in LaTeX by indenting the left and
right margins.  For example, \cite[pp. xiii]{Lamport86}:

\begin{quotation}
The LaTeX document preparation system is a special version of
Donald Knuth's TeX program.  TeX is a sophisticated program designed
to produce high-quality typesetting, especially for mathematical text.
{\ldots}

LaTeX represents a balance between functionality and ease of use.
Since I implemented most of it myself, there was also a continual
compromise between what I wanted to do and what I could do in a
reasonable amount of time.  {\ldots}


\end{quotation}






\subsection{Lists}

\markboth{The {\tt qwertz} Document Types}{Lists}
Three types of lists are supported, which differ according to the
type of label used to mark each item in the list.  Use {\tt itemize}
to create a list in which each item is marked with some symbol such as
a dash or bullet. The {\tt enum} tag is used to create an
enumeration, i.e. a list in which each item is labelled with a number
(or letter) indicating its rank or position in the list. Finally, use
{\tt descrip} to create a list in which each item is labelled by some
tag of your own choice. Lists of various types can nested.  For
example:

{\ldots}

\begin{itemize}
\item  A level one item.
\item  Here's level two 
\begin{enumerate}
\item  A level two item.
\item  Here's level three:
\begin{enumerate}
\item  A level three item.
\item  Here's level four:
\begin{description}
\item[Red.] \mbox{}

 Is the color of my true love's hair.
\item[Blue.] \mbox{}

 Is a property of some movies.
\item[Yellow.] \mbox{}

 Characterizes some forms of journalism.
\end{description}
\item A last level three item
\end{enumerate}
\item  A last level two item
\end{enumerate}
\item A last level one item.
\end{itemize}


{\ldots}




\subsection{Figures and Tables}

Encapsulated PostScript graphics can be created using a variety of
different editors.  If you are using Unix with an X11-based graphical
user-interface, you may want to try {\tt idraw}, which stores its
documents directly as Encapsulated PostScript files. Another
interesting X11-based drawing program is {\tt tgif}.

{\ldots}

Which would then appear as in figure \ref{issues}.

\begin{figure}[tbp]
\centerline{\epsffile{issues.ps}}
\caption{\label{issues}An {\tt idraw} Drawing }
\end{figure}





\subsection{Literate Programming}

The original motivation behind the development of these document
types was to create an environment for literate programming in an
arbitrary programming language similar to Donald Knuth's WEB system
for literate programming in Pascal \cite{Knuth84}.  {\ldots}

When formatted, spaces and line breaks are preserved:

\begin{verbatim}
main ()
{
        /* This is the famous hello world program */

        printf("hello world\n");
}
\end{verbatim}





\subsection{Mathematical Formulas}

The {\tt qwertz} document types include elements for describing
mathematical formulas completely within SGML, similar to the system
described in \cite{daphne89}.  {\ldots}

So, for example, 

\[         \sum_{i=1}^{n}x_{i} =         \int_{0}^{1}f \]

was typed as:
\begin{verbatim}
<dm>
        <sum><ll>i=1<ul>n<opd>x<inf>i</></sum> =
        <in><ll>0<ul>1<opd>f</in>
</dm>
\end{verbatim}


{\ldots}



\bibliographystyle{qwertz}
\bibliography{lit}
\end{document}
