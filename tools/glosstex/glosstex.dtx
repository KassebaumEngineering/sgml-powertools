% \iffalse -*- latex -*-
%
% $Id: glosstex.dtx,v 1.1.1.1 1997/08/22 04:02:16 jak Exp $
%
%   GlossTeX, a tool for the automatic preparation of glossaries.
%   Copyright (C) 1997 Volkan Yavuz
%
%   This program is free software; you can redistribute it and/or
%   modify it under the terms of the GNU General Public License
%   as published by the Free Software Foundation; either version 2
%   of the License, or (at your option) any later version.
%
%   This program is distributed in the hope that it will be useful,
%   but WITHOUT ANY WARRANTY; without even the implied warranty of
%   MERCHANTABILITY or FITNESS FOR A PARTICULAR PURPOSE.  See the
%   GNU General Public License for more details.
%
%   You should have received a copy of the GNU General Public License
%   along with this program; if not, write to the Free Software
%   Foundation, Inc., 675 Mass Ave, Cambridge, MA 02139, USA.
%
%   Volkan Yavuz, yavuzv@rumms.uni-mannheim.de
%
%% \CharacterTable
%%  {Upper-case    \A\B\C\D\E\F\G\H\I\J\K\L\M\N\O\P\Q\R\S\T\U\V\W\X\Y\Z
%%   Lower-case    \a\b\c\d\e\f\g\h\i\j\k\l\m\n\o\p\q\r\s\t\u\v\w\x\y\z
%%   Digits        \0\1\2\3\4\5\6\7\8\9
%%   Exclamation   \!     Double quote  \"     Hash (number) \#
%%   Dollar        \$     Percent       \%     Ampersand     \&
%%   Acute accent  \'     Left paren    \(     Right paren   \)
%%   Asterisk      \*     Plus          \+     Comma         \,
%%   Minus         \-     Point         \.     Solidus       \/
%%   Colon         \:     Semicolon     \;     Less than     \<
%%   Equals        \=     Greater than  \>     Question mark \?
%%   Commercial at \@     Left bracket  \[     Backslash     \\
%%   Right bracket \]     Circumflex    \^     Underscore    \_
%%   Grave accent  \`     Left brace    \{     Vertical bar  \|
%%   Right brace   \}     Tilde         \~}
%%
%
\def\filename{glosstex}
\def\fileversion{0.3}
\def\filedate{1997/06/10}
\let\docdate=\filedate
%
%<*driver>
\documentclass[draft]{ltxdoc}
\usepackage{array}
\usepackage{glosstex}

%\OnlyDescription
%\CodelineIndex
%\EnableCrossrefs

%\setlength{\arrayrulewidth}{0.4pt}
\setlength{\doublerulesep}{0pt}

\newcommand{\GloTeX}{Glo\TeX}
\newcommand{\MakeIndex}{\textsc{MakeIndex}}
\newcommand{\unix}{\textsc{Unix}}
\newcommand{\package}[1]{\textsf{#1}}
\newcommand{\file}[1]{\texttt{#1}}
\newcommand{\opt}[1]{$[$#1$]$}

\newenvironment{definition}{\begin{quote}}{\end{quote}}
\newenvironment{example}{\begin{quote}}{\end{quote}}

\begin{document}
\DocInput{glosstex.dtx}
\end{document}
%</driver>
%
% \fi
%
% \CheckSum{740}
%
% \title{\GlossTeX{} \fileversion}
% \author{\scshape Volkan Yavuz\footnote{e-mail: yavuzv@rumms.uni-mannheim.de}}
% \date{\filedate}
% \maketitle
%
% \begin{abstract}
%  \GlossTeX{} is a tool for the preparation of glossaries, lists of
%  acronyms or sorted lists in general. It greatly
%  simplifies this task. One or more glossary-definition files serve as
%  databases which contain descriptions of terms. These terms are
%  identified through labels. Based upon labels set into the
%  \TeX{}-source, \GlossTeX{} determines which entries have to appear in
%  the typeset list. \GlossTeX{} uses \MakeIndex{} for the sorting of the
%  lists. References to the place where a term appears in the text
%  can be set in the list. A term consists of a label which is
%  used to identify it, an optional item describing the typeset
%  output, an optional longform and the actual text representing
%  it. There are many ways to access each of these fields within
%  the document. It is also possible to generate cross-references 
%  to another term.
% \end{abstract}
%
% \tableofcontents
% \listoftables
% \printglosstex(acr)
%
% \section{Introduction}
% \subsection{Purpose}
% \GlossTeX{} is a tool for the automatic preparation of glossaries,
% lists of acronyms, nomenclature and sorted lists in general. Based upon the
% labels set into the \TeX{}-source, \GlossTeX{} determines which entries
% from a glossary-definition file have to be processed to generate the
% list. \GlossTeX{} then
% creates an intermediate file that has to be processed by \MakeIndex{} for
% sorting. The output of \MakeIndex{}  is then included into the
% \TeX{}-source for typesetting. This whole process can be automated by
% using a makefile. With each term associated is an item representing
% the typeset output, an optional long-form if it's an acronym and an
% optional descriptive text. These elements can all be accessed
% within the document in many ways.
%
% \subsection{History}
% I created \GlossTeX{} because there were no tools for the preparation
% of glossaries that fit my needs. \GlossTeX{} is mainly a combination of
% the features of the packages \package{acronym}, \package{nomencl}
% and \GloTeX{}. The commands starting with |\ac| are taken more or
% less from \package{acronym} and the way \GlossTeX{} 
% handles page references is almost identical to the way 
% \package{nomencl} does. The use of glossary databases is inspired by
% \GloTeX{}. 
%
% \subsection{Legalese}
% \GlossTeX{} is provided ``as is'' and comes with absolutely no
% warranty. It is covered by the GNU General Public License (see the
% file |COPYING| that comes with this package).
%
% \begin{center}
%   \noindent\copyright\ Volkan Yavuz, 1997
% \end{center}
%
% \section{Usage}
% The \LaTeX{}-macros needed
% by \GlossTeX{} have to be included into the source using
%
% \begin{example}
%   |\usepackage|\opt{|[|\meta{options}|]|}|{glosstex}|
% \end{example}
%
% where \meta{options} may be one of |refpage| or |norefpage|. The
% latter one is the default. These options are described later in
% section~\ref{sec:page_references}.
%
% Whenever you want a term to appear in the glossary, you insert
% \DescribeMacro{\glosstex}
% \begin{definition}
%   |\glosstex|\opt{|(|\meta{list}|)|}\opt{|[|\meta{pageref-mode}|]|}|{|\meta{label}|}|
% \end{definition}
%
% into the text. \meta{label} references the entry and the
% optional arguments \meta{pageref-mode} and \meta{list} determine the
% mode for page 
% references and the list the referenced term should appear in.
% \meta{pageref-mode} works in conjunction with the optional arguments
% |refpage| and
% |norefpage| to the package and defaults to |[p]|. \meta{list} defaults
% to |(glo)|. 
%
% But you can create $n$ lists and make each \meta{label} appear in
% any \meta{list}, each appearance independent of the other ones. To 
% faciliate the creation of multiple \meta{list}s and especially make it easy
% to achieve the most likely usage, there are actually two sets of
% commands. One set starting with |\gl| defaults to \meta{list}
% |(glo)| (glossary) and \meta{pageref-mode} |[p]|, the other set starting with
% |\ac| defaults to \meta{list} |(acr)| (list of acronyms) and
% \meta{pageref-mode} |[n]|. The following is the equivalent to |\glosstex|:
%
% \DescribeMacro{\acronym}
% \begin{definition}
%   |\acronym|\opt{|(|\meta{list}|)|}\opt{|[|\meta{pageref-mode}|]|}|{|\meta{label}|}| 
% \end{definition}
%
% These two commands can also be called as |\glosstex{*}| and
% |\acronym{*}| to include all term found in the |.gdf|-files into the
% corresponding \meta{list}. This is similar to the command
% |\nocite{*}| in \BibTeX{}. But do \emph{not} use something like
% |\gls{*}| (see below).
%
% There is another set of commands that produce both an entry in the
% list (optional) and typeset output. 
%
% \DescribeMacro{\gls}
% \DescribeMacro{\gls*}
% \begin{definition}
%   |\gls|\opt{|*|}\opt{|(|\meta{list}|)|}\opt{|[|\meta{pageref-mode}|]|}|{|\meta{label}|}|
% \end{definition}
%
% |\gls| typesets \meta{item} and produces a list entry (\meta{list}
% defaults to |(glo)|). 
% The starred version |\gls*| just produces typeset output
% without a list-entry. (To be honest, a list-entry is produced, but
% it never appears in the output.) The starred version implicitly produces entries
% with \meta{list-mode} |[n]| (never) while the unstarred versions set
% \meta{list-mode} to |[a]| (always). There is no other way of
% specifying the \meta{list-mode} of an entry.
%
% \DescribeMacro{\ac}
% \DescribeMacro{\acs}
% \DescribeMacro{\acl}
% \DescribeMacro{\acf}
% \DescribeMacro{\ac*}
% \DescribeMacro{\acs*}
% \DescribeMacro{\acl*}
% \DescribeMacro{\acf*}
% \begin{definition}
%   |\ac|\opt{|*|}\opt{|(|\meta{list}|)|}\opt{|[|\meta{pageref-mode}|]|}|{|\meta{label}|}|\\
%   |\acs|\opt{|*|}\opt{|(|\meta{list}|)|}\opt{|[|\meta{pageref-mode}|]|}|{|\meta{label}|}|\\
%   |\acl|\opt{|*|}\opt{|(|\meta{list}|)|}\opt{|[|\meta{pageref-mode}|]|}|{|\meta{label}|}|\\
%   |\acf|\opt{|*|}\opt{|(|\meta{list}|)|}\opt{|[|\meta{pageref-mode}|]|}|{|\meta{label}|}|
% \end{definition}
%
% These all produce typeset output. |\acs| typesets
% \meta{item}, |\acl| typesets \meta{long-form} and |\acf|
% typesets \meta{item} |(|\meta{long-form}|)|. |\ac|
% works like |\acf| at it's first invocation and like |\acs| on all
% subsequent ones. Using |\ac| you can make sure that an acronym is
% always spelled out at least once in your document while safely using
% the short form whenever possible. The starred versions |\ac*|,
% |\acs*|, |\acl*| and |\acf*| just produce typeset
% output. Table~\ref{tab:commands} gives an overview of all this.
%
% \begin{table}[htbp]
%  \begin{center}
%    \begin{tabular}{rllll}
%      \hline\hline
%      defaults & only list & both & only text & output\\
%      \hline
%      |(glo)[p]| & |\glosstex| & |\gls| & |\gls*| & \meta{item}\\
%      |(acr)[n]| & |\acronym| & |\ac| & |\ac*| 
% & \meta{item} or \meta{item} |(|\meta{long-form}|)|\\
%      |(acr)[n]| & & |\acs| & |\acs*| & \meta{item}\\
%      |(acr)[n]| & & |\acl| & |\acl*| & \meta{long-form}\\
%      |(acr)[n]| & & |\acf| & |\acf*| & \meta{item} |(|\meta{long-form}|)|\\
%      \hline\hline
%    \end{tabular}
%    \caption{Overview of the different sets of commands.}
%    \label{tab:commands}
%  \end{center}
% \end{table}
%
% \subsection{The glossary definition file}
% A glossary-definition file\glosstex{gdf-file} (suffix |.gdf|) is needed which serves
% as a database for \GlossTeX{}, holding the actual descriptions of all
% terms. You can have $m$ |.gdf|-files that contain the definitions to the
% \meta{label}s you reference in your documents. Entries have the form
% \begin{definition}
%   |@entry{|\meta{label}\opt{|, |\meta{item}\opt{|, |\meta{long-form}}}|}| \opt{\meta{text}}
% \end{definition}
%
% where \meta{label} is used to identify the entry and \meta{text} may contain
% any amount of \TeX{}-source, being the actual definition of the
% item. You should know that \meta{label} is used to construct
% \TeX{}-macros, so it shouldn't contain funny characters or you will
% most likely get funny errors.
% The optional argument \meta{item} describes the appearance of the item
% in the produced list. If omitted, it defaults to \meta{label}. It can
% be used when some special form of typesetting is wanted. \meta{item}
% can contain any \TeX{}-construct, as long as each ``|{|'' has a
% corresponding ``|}|''. Otherwise, \GlossTeX{} will get confused. The
% same applies to \meta{long-form}, except that it defaults into an
% empty string if not specified.
%
% This is the |.gdf|-file used for this documentation. Note the use of
% ``|~|'' as the quote-character. Also note that all lines
% until the first line starting with |@entry{| are
% ignored. Additionally, all lines starting with ``|%|'' are ignored,
% too. Thus they
% can serve as comments.
%    \begin{macrocode}
%<*gdf>
% -*- latex -*-

This is a database file for GlossTeX.

@entry{mst-file, \texttt{.mst}-file} Style file for \MakeIndex{},
describing the input and output format of read and written files.

@entry{gdf-file, \texttt{.gdf}-file} This file is the database file
containing definitions for Gloss\TeX{}.

@entry{gxs-file, \texttt{.gxs}-file} Intermediate file produced by
\GlossTeX{} to be processed by \MakeIndex{}.

@entry{glx-file, \texttt{.glx}-file} This file contains the sorted
lists, ready to be read by \LaTeX{}.

@entry{gxg-file, \texttt{.gxg}-file} This is the log-file produced by
the \GlossTeX{}-run. See also \glxref{glg-file}.

@entry{glg-file, \texttt{.glg}-file} This is the log-file produced by
the \MakeIndex{}-run. See also \glxref{gxg-file}.

@entry{ASCII, ASCII, American Standard Code for Information Interchange} 
A character encoding. See also \glxref{EBCDIC}.

@entry{EBCDIC, EBCDIC, Extended Binary Coded Decimals Interchange Code} 
A character encoding exclusively used on mainframes. See also
\glxref{ASCII}.
%</gdf>
%    \end{macrocode}
%
% \subsection{Invocation}
% After the first run of \LaTeX{}, the |.aux|-file contains all
% necessary information for the preparation of the glossary. \GlossTeX\
% is then invoked to read one or more |.gdf|-files and output all
% definitions that are referenced in the |.aux|-file. The output of
% \GlossTeX{} is then processed by \MakeIndex{} for sorting.
%
% \GlossTeX{} is invoked in a \unix{}-like environment using the
% following command
%
%\begin{example}
%  |glosstex |\meta{aux-file}| |\meta{gdf-file}| |\opt{\meta{gdf-file}\opt{...}}| |\opt{|-v|\opt{|0|\ldots|5|}} 
% \end{example}
%
% This produces 2 files as output, one \gls{gxs-file} to be input into
% \MakeIndex{} and a log-file with extension |.gxg|\glosstex{gxg-file} which contains
% more detailed information. The |-v| option selects how verbose
% \GlossTeX{} should be when writing the log-file. |-v| is equal to
% |-v4| and |-v2| is the default. |-v0| makes \GlossTeX{} shut his
% mouth and only report errors and |-v5| makes \GlossTeX{} really talkative.
%
% \MakeIndex{}  has to be invoked in this way
%
% \begin{example}
%   |makeindex |\meta{gxs-file}| -o |\meta{glx-file}| -s |\meta{ist-file}| |\opt{|-t |\meta{glg-file}}
% \end{example}
%
% The commands
%
% \begin{example}
% |glosstex thesis thesis.gdf master.gdf|\\
% |makeindex thesis.gxs -o thesis.glx -s glosstex.mst|
% \end{example}
%
% produce the final \gls{glx-file} which is then included by
% \DescribeMacro{\printglosstex}
% \begin{example}
%  |\printglosstex|\opt{|(|\meta{list}|)|}\opt{|[|\meta{pageref-mode}|]|}
% \end{example}
% during the next \LaTeX{}-run. The argument \meta{pageref-mode}
% supercedes the one given to the entries individually for each
% \meta{list} and defaults to |[p]|. You can turn on page references
% unconditionally for each \meta{list} individually by using
% \meta{pageref-mode} |[a]| and turn it off by using \meta{pageref-mode}
% |[n]| as argument to |\printglosstex|. See also
% table~\ref{tab:references} for an overview of these options.
% 
% Note that, no matter of how many 
% \meta{list}s you produce, there is always exactly \emph{one}
% |.glx|-file which contains the entries of \emph{all} produced
% lists. Depending on whether you use \meta{item} or \meta{long-form}
% in your text or you have cross-references, it may be necessary to
% run \LaTeX{} and \GlossTeX{} up to 4 times until all references are
% resolved. Watch out for warnings from \package{glosstex} during a
% \LaTeX{} run. Messages about unresolved \meta{label}s from
% \GlossTeX{} are caused by missing definitions in the |.gdf|-files.
%
% \subsection{Page References}
% \label{sec:page_references}
% You may want a reference in the list to the place where the term
% first appears in the text. This can be done using the optional
% argument \meta{pageref-mode}. That argument in combination with the option
% to |\usepackage| controls these references. Table~\ref{tab:references} gives
% an overview of all possible combinations of these 3 arguments. A
% ``$\times$'' indicates that a reference is produced.
%
% \begin{table}[htbp]
% \begin{center}
%  \begin{tabular}{l|ccc|ccc}
%    \hline\hline
%    \hfill list & & |refpage| & & & |norefpage|\\
%    entry & |a| & |p| & |n| & |a| & |p| & |n|\\
%    \hline
%    |a| & $\times$ & $\times$ & $-$ & $\times$ & $\times$ & $-$\\
%    |p| & $\times$ & $\times$ & $-$ & $\times$ & $-$ & $-$\\
%    |n| & $\times$ & $-$ & $-$ & $\times$ & $-$ & $-$\\
%    \hline\hline
%  \end{tabular}
%  \caption{Options controlling the appearance of page references.}
%  \label{tab:references}
% \end{center}
% \end{table}
%
% One possible usage of this feature: while debugging a document, turn on
% page references by using the option |refpage| to the package. Every
% entry included with the modes |[a]| (always) or the default |[p]|
% (package) will contain a reference. After debugging, remove the option
% |refpage| and only those entries that were included with mode |[a]| will
% still have a reference.
%
% \subsection{Cross-References}
% \DescribeMacro{\glxref}
% \DescribeMacro{\glxref*}
% It may be useful to use cross-references in entries. Assume
% you have referenced |\glosstex{ascii}| which describes the term
% \ac{ASCII}. You may also want to include \ac*{EBCDIC} as an
% example for another character encoding. To achieve this, write this
% into the definition of \ac{ASCII}
%
% \begin{example}
% |See also \glxref{ebcdic}.|
% \end{example}
%
% and \GlossTeX{} then produces ``See also EBCDIC'' and also includes
% the definition for \ac*{EBCDIC} into the same list \ac{ASCII}
% appears in. Note 
% that |\glxref| is only available within the \meta{text} argument in
% the |.gdf|-file since it only makes sense within a
% \meta{list}. There is also a starred version |\glxref*| that doesn't
% produce typeset output.
%
% \subsection{\GlossTeX{} and \package{nomencl}}
% It is possible to use \package{nomencl} and \GlossTeX{} in one
% document without problems. 
%
% The following commands show how to deal with documents using both
% \GlossTeX{} and \textsf{nomencl}.
% \begin{example}
% |latex thesis|\\
% |glosstex thesis thesis.gdf |\\
% |makeindex thesis.gxs -o thesis.glx -s glosstex.mst|\\
% |makeindex thesis.glo -o thesis.gls -s nomencl.ist|\\
% |latex thesis|
% \end{example}
%
% \section{Customizing}
% \GlossTeX{} can be customized by using the file |glosstex.cfg| which
% is automatically loaded if it is present. The file |glosstex.std| is
% the default configuration file that is mandatory to \GlossTeX{}. It
% shows all aspects that are meant to be customized, so let's discuss
% it now.
%
% Each term that gets typeset either goes through |\GLX@output@short|
% or |\GLX@output@long|, depending whether it's the \meta{item} or
% \meta{long-form}. These macros each take 3 arguments and get called
% this way:
% \begin{definition}
%   |\GLX@output@short{|\meta{label}|}{|\meta{list}|}{|\meta{item}|}|\\
%   |\GLX@output@long{|\meta{label}|}{|\meta{list}|}{|\meta{long-form}|}|
% \end{definition}
%
% \begin{macro}{\GLX@output@short}
% \begin{macro}{\GLX@output@long}
% In |\GLX@output@short| we test whethe we are in the list of acronyms
% (|GLX@acdef@list|) and typeset \meta{item} with caps and small caps
% if we are. Otherwise, just typeset \meta{item}. An index-entry is
% also produces, using \meta{label} as the sort-key.
%    \begin{macrocode}
%<*std>
\newcommand{\GLX@output@short}[3]{%
  \ifthenelse{\equal{#2}{\GLX@acdef@list}}{\textsc{#3}}{#3}%
  \index{#1@#3}} 
\newcommand{\GLX@output@long}[3]{#3}
%    \end{macrocode}
% \end{macro}
% \end{macro}
%
% \begin{macro}{\glosstexpage}
% This is used to typeset the page at the end of a definition. It uses
% |\pagename| so that should be defined elsewhere.
%    \begin{macrocode}
\newcommand{\glosstexpage}[1]{\nobreak\hfill\emph{\pagename~#1}\nobreak}
%    \end{macrocode}
% \end{macro}
%
% \DescribeMacro{\glxgldefault}
% \DescribeMacro{\glxacdefault}
% These macros set the defaults for \meta{list} and
% \meta{pageref-mode} which the commands starting with |\gl| (except for
% |\glxref|, of course) and |\ac| are
% using.
%    \begin{macrocode}
\glxgldefault{glo}{p}
\glxacdefault{acr}{n}
%    \end{macrocode}
%
% \begin{macro}{\GLX@benv@glo}
% \begin{macro}{\GLX@eenv@glo}
% \begin{macro}{\GLX@item@glo}
% \begin{macro}{\GLX@benv@acr}
% \begin{macro}{\GLX@eenv@acr}
% \begin{macro}{\GLX@item@acr}
% Each \meta{list} gets embedded into |\GLX@benv@|\meta{list} and
% |\GLX@eenv@|\meta{list}, so these macros should provide a reasonable
% environment. Each line itself is typeset using
% |\GLX@item@|\meta{list} which gets called with 7 arguments.
% \begin{definition}
% |\GLX@item@|\meta{list}|{|\meta{label}|}{|\meta{item}|}{|\meta{long-form}|}{|\meta{text}|}{|\meta{list}|}{|\meta{list-mode}|}{|\meta{page-stuff}|}| 
% \end{definition}
%    \begin{macrocode}
\newcommand{\GLX@benv@glo}{\begin{description}}
\newcommand{\GLX@eenv@glo}{\end{description}}
\newcommand{\GLX@item@glo}[7]{%
  \item[#2]\ifx#3\empty\else\emph{#3}\space\fi#4\space#7}

\newcommand{\GLX@item@acr@label}[1]{\mbox{#1}\dotfill}
\newcommand{\GLX@benv@acr}{
  \begin{list}{}{%
      \renewcommand{\makelabel}{\GLX@item@acr@label}%
      \setlength{\labelwidth}{7em}%
      \leftmargin\labelwidth \advance\leftmargin by \labelsep}}
\newcommand{\GLX@eenv@acr}{\end{list}}
\newcommand{\GLX@item@acr}[7]{%
  \item[\textsc{#2}]%
    \ifx#3\empty\else#3\quad\fi\ifx#4\empty\else#4\fi\space#7}
%    \end{macrocode}
% \end{macro}
% \end{macro}
% \end{macro}
% \end{macro}
% \end{macro}
% \end{macro}
%
% \begin{macro}{\glossaryname}
% \begin{macro}{\listacronymname}
% \begin{macro}{\pagename}
% These are defined to contain some default strings if they're not
% already defined (|\pagename| e.g.\ is defined trough \package{babel}).
%    \begin{macrocode}
\ifx\glossaryname\@undefined
  \def\glossaryname{Glossary}
\fi
\ifx\listacronymname\@undefined
  \def\listacronymname{List of Acronyms}
\fi
\ifx\pagename\@undefined
  \def\pagename{page}
\fi
%    \end{macrocode}
% \end{macro}
% \end{macro}
% \end{macro}
%
% \DescribeMacro{\glxheading}
% Each \meta{list} starts with an appropriate heading which is defined
% by 
% \begin{definition}
%   |\glxheading[|\meta{list}|]{|\meta{definition}|}|.
% \end{definition}
%    \begin{macrocode}
\ifx\chapter\@undefined
  \glxheading{\section*{\glossaryname}}
  \glxheading[acr]{\section*{\listacronymname}}
\else
  \glxheading{\chapter*{\glossaryname}}
  \glxheading[acr]{\chapter*{\listacronymname}}
\fi
%</std>
%    \end{macrocode}
%
% \section{Some Details}
% While reading the |.aux|-file, \GlossTeX{} only considers the first
% appearance of one \meta{item} for each \meta{list}. All subsequent
%  entries are silently ignored. 
% (Almost silently, because the |.gxg|-file will contain detailed
% information about this, and more.) But if the first entry says 
% not to produce a page reference and a following one says to do so, then
% the latter will supercede the former. The same applies if a
% term is referenced without the option not to generate a list-entry
% (all commands containing a |*|, e.g.\ |\gl*|), but a following tells
% to do so.
%
% While reading one or more |.gdf|-files, only the first definition is
% used, all other entries are ignored. This fact can be utilised
% in some way. Assume you have a |master.gdf| which contains general
% terms and a file |thesis.gdf| which only contains terms that are
% intended for use in your thesis. Whenever an entry is present in both
% |.gdf|-files, the one from |thesis.gdf| should be taken. To achieve
% this, specify |thesis.gdf| \emph{before} |master.gdf|.
%
% The |.gxg|-file contains additional information, e.\,g.~when no entry
% was found for a label.
%
% After \GlossTeX{} is finished, it will print some statistics about read
% labels, unresolved entries and the like. More detailed information can
% then be found in the |.gxg|-file.
%
% \section{Portability}
%
% \GlossTeX{} should compile in any ANSI C environment. I have built
% \GlossTeX{} on Linux 2.0.0 with gcc 2.7.2, GNU make 3.7.4 and libc
% 5.2.17. It also compiled out-of-the box on DOS using DJGPP
% 2.6.3. There is also an OS/2 port on CTAN in
% |support/|\-|glosstex/|\-|binaries/|\-|os2|. If 
% you have successfully ported \GlossTeX{} on some other platform (like VMS
% or Macintosh) please contact me, so I can enhance \GlossTeX{}.
%
% Additionally, see the file |TODO| in this package for
% known bugs (also called features) and not yet implemented features
% (also called bugs).
%
% \section{Acknowledgments}
% I would like to thank these people who have contributed to the
%  development of \GlossTeX{}: 
%
% {\scshape
% Stefan A.~Deutscher, Michael Friendly
% }
%
% \StopEventually{\printglosstex[a]}
% 
% \section{The implementation}
% \subsection{The package \package{glosstex.sty}}
%
% First, we identify ourselves.
%    \begin{macrocode}
%<*package>
\NeedsTeXFormat{LaTeX2e}[1996/12/01] 
\ProvidesPackage{glosstex}
  [\filedate\space\fileversion\space GlossTeX package]
%    \end{macrocode}
%
% The \package{ifthen} package is needed for some |\ifthenelse|-constructs.
%    \begin{macrocode}
\RequirePackage{ifthen}
\newif\if@GLX@pagerefmode
%    \end{macrocode}
%
% \begin{macro}{\GlossTeX}
% This is used to typeset a nice logo for this package, which looks
% like ``\GlossTeX{}''.
%    \begin{macrocode}
\newcommand{\GlossTeX}{Gloss\TeX}
%    \end{macrocode}
% \end{macro}
%
% \begin{macro}{\GlossTeXPage}
% Call |\glosstexpage| depending on |#2| (\meta{pageref-mode}),
% |\if@GLX@pagerefmode| and |\GLX@theglosstex@pagerefmode|.
%    \begin{macrocode}
\newcommand{\GlossTeXPage}[2]{\ignorespaces%
  \ifthenelse{\boolean{@GLX@pagerefmode}}%
    {% refpage true
      \ifthenelse{%
        \equal{\GLX@theglosstex@pagerefmode}{a}}{\glosstexpage{#2}}{%
        \ifthenelse{\equal{\GLX@theglosstex@pagerefmode}{n}}{}{%
          \ifthenelse{\equal{#1}{a}}{\glosstexpage{#2}}{}%
          \ifthenelse{\equal{#1}{p}}{\glosstexpage{#2}}{}%
          \ifthenelse{\equal{#1}{n}}{}{}}}}%
    {% refpage false
      \ifthenelse{%
        \equal{\GLX@theglosstex@pagerefmode}{a}}{\glosstexpage{#2}}{%
        \ifthenelse{\equal{\GLX@theglosstex@pagerefmode}{n}}{}{%
          \ifthenelse{\equal{#1}{a}}{\glosstexpage{#2}}{}%
          \ifthenelse{\equal{#1}{p}}{}{}%
          \ifthenelse{\equal{#1}{n}}{}{}}}}%
    }%
%    \end{macrocode}
% \end{macro}
%
% Now declare the options this package accepts and execute the
% standard option |norefpage|.
%    \begin{macrocode}
\DeclareOption{refpage}{\@GLX@pagerefmodetrue}
\DeclareOption{norefpage}{\@GLX@pagerefmodefalse}
\ExecuteOptions{norefpage}
\ProcessOptions
%    \end{macrocode}
%
% \begin{macro}{\GlossTeXNull}
% This one is used as a dummy to work around a feature in \MakeIndex{}.
%    \begin{macrocode}
\newcommand{\GlossTeXNull}[1]{}
%    \end{macrocode}
% \end{macro}
%
% \begin{macro}{\glosstex}
% |\glosstex|\opt{|(|\meta{list}|)|}\opt{|[|\meta{pageref-mode}|]|}|{|\meta{label}|}| 
% just produces a list-entry.
%    \begin{macrocode}
\DeclareRobustCommand\glosstex{\GLX@glosstex}
\newcommand{\GLX@glosstex}{%
  \@ifnextchar[{\GLX@@glosstexs}{%
    \@ifnextchar({\GLX@@glosstexr}
    {\GLX@@@glosstexs[\GLX@gldef@page](\GLX@gldef@list)}}}
\newcommand{\GLX@@glosstexs}{}
\def\GLX@@glosstexs[#1]{%
  \@ifnextchar({\GLX@@@glosstexs[#1]}
  {\GLX@@@glosstexs[#1](\GLX@gldef@list)}}
\newcommand{\GLX@@glosstexr}{}
\def\GLX@@glosstexr(#1){%
  \@ifnextchar[{\GLX@@@glosstexr(#1)}
  {\GLX@@@glosstexs[\GLX@gldef@page](#1)}}
\newcommand{\GLX@@@glosstexs}{}
\def\GLX@@@glosstexs[#1](#2)#3{\GLX@do@glosstex{#3}{#1}{#2}{a}}
\newcommand{\GLX@@@glosstexr}{}
\def\GLX@@@glosstexr(#1)[#2]#3{\GLX@do@glosstex{#3}{#2}{#1}{a}}
%    \end{macrocode}
% \end{macro}
%
% \begin{macro}{\acronym}
% |\acronym|\opt{|(|\meta{list}|)|}\opt{|[|\meta{pageref-mode}|]|}|{|\meta{label}|}| 
% just produces a list-entry.
%    \begin{macrocode}
\DeclareRobustCommand\acronym{\GLX@acronym}
\newcommand{\GLX@acronym}{%
  \@ifnextchar[{\GLX@@acronyms}{%
    \@ifnextchar({\GLX@@acronymr}{\GLX@@@acronyms[n](acr)}}}
\newcommand{\GLX@@acronyms}{}
\def\GLX@@acronyms[#1]{%
  \@ifnextchar({\GLX@@@acronyms[#1]}{\GLX@@@acronyms[#1](acr)}}
\newcommand{\GLX@@acronymr}{}
\def\GLX@@acronymr(#1){%
  \@ifnextchar[{\GLX@@@acronymr(#1)}{\GLX@@@acronyms[n](#1)}}
\newcommand{\GLX@@@acronyms}{}
\def\GLX@@@acronyms[#1](#2)#3{\GLX@do@glosstex{#3}{#1}{#2}{a}}
\newcommand{\GLX@@@acronymr}{}
\def\GLX@@@acronymr(#1)[#2]#3{\GLX@do@glosstex{#3}{#2}{#1}{a}}
%    \end{macrocode}
% \end{macro}
%
% \begin{macro}{\gls}
% \begin{macro}{\gls*}
%    \begin{macrocode}
\DeclareRobustCommand\gls{%
  \@ifstar{\GLX@glsS}{\GLX@gls}}
\newcommand{\GLX@gls}{%
  \@ifnextchar[{\GLX@@glss}{%
    \@ifnextchar({\GLX@@glsr}
    {\GLX@@@glss[\GLX@gldef@page](\GLX@gldef@list)}}}
\newcommand{\GLX@@glss}{}
\def\GLX@@glss[#1]{%
  \@ifnextchar({\GLX@@@glss[#1]}{\GLX@@@glss[#1](\GLX@gldef@list)}}
\newcommand{\GLX@@glsr}{}
\def\GLX@@glsr(#1){%
  \@ifnextchar[{\GLX@@@glsr(#1)}{\GLX@@@glss[\GLX@gldef@page](#1)}}
\newcommand{\GLX@@@glss}{}
\def\GLX@@@glss[#1](#2)#3{%
  \GLX@do@glosstex{#3}{#1}{#2}{a}\GLX@do@acs{#3}{#2}}
\newcommand{\GLX@@@glsr}{}
\def\GLX@@@glsr(#1)[#2]#3{%
  \GLX@do@glosstex{#3}{#2}{#1}{a}\GLX@do@acs{#3}{#1}}
\newcommand{\GLX@glsS}{%
  \@ifnextchar[{\GLX@@glsSs}{%
    \@ifnextchar({\GLX@@glsSr}
    {\GLX@@@glsSs[\GLX@gldef@page](\GLX@gldef@list)}}}
\newcommand{\GLX@@glsSs}{}
\def\GLX@@glsSs[#1]{%
  \@ifnextchar({\GLX@@@glsSs[#1]}{\GLX@@@glsSs[#1](\GLX@gldef@list)}}
\newcommand{\GLX@@glsSr}{}
\def\GLX@@glsSr(#1){%
  \@ifnextchar[{\GLX@@@glsSr(#1)}{\GLX@@@glsSs[\GLX@gldef@page](#1)}}
\newcommand{\GLX@@@glsSs}{}
\def\GLX@@@glsSs[#1](#2)#3{%
  \GLX@do@glosstex{#3}{#1}{#2}{n}\GLX@do@acs{#3}{#2}}
\newcommand{\GLX@@@glsSr}{}
\def\GLX@@@glsSr(#1)[#2]#3{%
  \GLX@do@glosstex{#3}{#2}{#1}{n}\GLX@do@acs{#3}{#1}}
%    \end{macrocode}
% \end{macro}
% \end{macro}
%
% \begin{macro}{\ac}
% \begin{macro}{\ac*}
%    \begin{macrocode}
\DeclareRobustCommand\ac{%
  \@ifstar{\GLX@acS}{\GLX@ac}}
\newcommand{\GLX@ac}{%
  \@ifnextchar[{\GLX@@acs}{%
    \@ifnextchar({\GLX@@acr}
    {\GLX@@@acs[\GLX@acdef@page](\GLX@acdef@list)}}}
\newcommand{\GLX@@acs}{}
\def\GLX@@acs[#1]{%
  \@ifnextchar({\GLX@@@acs[#1]}{\GLX@@@acs[#1](\GLX@acdef@list)}}
\newcommand{\GLX@@acr}{}
\def\GLX@@acr(#1){%
  \@ifnextchar[{\GLX@@@acr(#1)}{\GLX@@@acs[\GLX@acdef@page](#1)}}
\newcommand{\GLX@@@acs}{}
\def\GLX@@@acs[#1](#2)#3{%
  \GLX@do@glosstex{#3}{#1}{#2}{a}\GLX@do@ac{#3}{#2}}
\newcommand{\GLX@@@acr}{}
\def\GLX@@@acr(#1)[#2]#3{%
  \GLX@do@glosstex{#3}{#2}{#1}{a}\GLX@do@ac{#3}{#1}}
\newcommand{\GLX@acS}{%
  \@ifnextchar[{\GLX@@acSs}{%
    \@ifnextchar({\GLX@@acSr}
    {\GLX@@@acSs[\GLX@acdef@page](\GLX@acdef@list)}}}
\newcommand{\GLX@@acSs}{}
\def\GLX@@acSs[#1]{%
  \@ifnextchar({\GLX@@@acSs[#1]}{\GLX@@@acSs[#1](\GLX@acdef@list)}}
\newcommand{\GLX@@acSr}{}
\def\GLX@@acSr(#1){%
  \@ifnextchar[{\GLX@@@acSr(#1)}{\GLX@@@acSs[\GLX@acdef@page](#1)}}
\newcommand{\GLX@@@acSs}{}
\def\GLX@@@acSs[#1](#2)#3{%
  \GLX@do@glosstex{#3}{#1}{#2}{n}\GLX@do@ac{#3}{#2}}
\newcommand{\GLX@@@acSr}{}
\def\GLX@@@acSr(#1)[#2]#3{%
  \GLX@do@glosstex{#3}{#2}{#1}{n}\GLX@do@ac{#3}{#1}}
%    \end{macrocode}
% \end{macro}
% \end{macro}
%
% \begin{macro}{\acs}
% \begin{macro}{\acs*}
%    \begin{macrocode}
\DeclareRobustCommand\acs{%
  \@ifstar{\GLX@acsS}{\GLX@acs}}
\newcommand{\GLX@acs}{%
  \@ifnextchar[{\GLX@@acss}{%
    \@ifnextchar({\GLX@@acsr}
    {\GLX@@@acss[\GLX@acdef@page](\GLX@acdef@list)}}}
\newcommand{\GLX@@acss}{}
\def\GLX@@acss[#1]{%
  \@ifnextchar({\GLX@@@acss[#1]}{\GLX@@@acss[#1](\GLX@acdef@list)}}
\newcommand{\GLX@@acsr}{}
\def\GLX@@acsr(#1){%
  \@ifnextchar[{\GLX@@@acsr(#1)}{\GLX@@@acss[\GLX@acdef@page](#1)}}
\newcommand{\GLX@@@acss}{}
\def\GLX@@@acss[#1](#2)#3{%
  \GLX@do@glosstex{#3}{#1}{#2}{a}\GLX@do@acs{#3}{#2}}
\newcommand{\GLX@@@acsr}{}
\def\GLX@@@acsr(#1)[#2]#3{%
  \GLX@do@glosstex{#3}{#2}{#1}{a}\GLX@do@acs{#3}{#1}}
\newcommand{\GLX@acsS}{%
  \@ifnextchar[{\GLX@@acsSs}{%
    \@ifnextchar({\GLX@@acsSr}
    {\GLX@@@acsSs[\GLX@acdef@page](\GLX@acdef@list)}}}
\newcommand{\GLX@@acsSs}{}
\def\GLX@@acsSs[#1]{%
  \@ifnextchar({\GLX@@@acsSs[#1]}{\GLX@@@acsSs[#1](\GLX@acdef@list)}}
\newcommand{\GLX@@acsSr}{}
\def\GLX@@acsSr(#1){%
  \@ifnextchar[{\GLX@@@acsSr(#1)}{\GLX@@@acsSs[\GLX@acdef@page](#1)}}
\newcommand{\GLX@@@acsSs}{}
\def\GLX@@@acsSs[#1](#2)#3{%
  \GLX@do@glosstex{#3}{#1}{#2}{n}\GLX@do@acs{#3}{#2}}
\newcommand{\GLX@@@acsSr}{}
\def\GLX@@@acsSr(#1)[#2]#3{%
  \GLX@do@glosstex{#3}{#2}{#1}{n}\GLX@do@acs{#3}{#1}}
%    \end{macrocode}
% \end{macro}
% \end{macro}
%
% \begin{macro}{\acl}
% \begin{macro}{\acl*}
%    \begin{macrocode}
\DeclareRobustCommand\acl{%
  \@ifstar{\GLX@aclS}{\GLX@acl}}
\newcommand{\GLX@acl}{%
  \@ifnextchar[{\GLX@@acls}{%
    \@ifnextchar({\GLX@@aclr}
    {\GLX@@@acls[\GLX@acdef@page](\GLX@acdef@list)}}}
\newcommand{\GLX@@acls}{}
\def\GLX@@acls[#1]{%
  \@ifnextchar({\GLX@@@acls[#1]}{\GLX@@@acls[#1](\GLX@acdef@list)}}
\newcommand{\GLX@@aclr}{}
\def\GLX@@aclr(#1){%
  \@ifnextchar[{\GLX@@@aclr(#1)}{\GLX@@@acls[\GLX@acdef@page](#1)}}
\newcommand{\GLX@@@acls}{}
\def\GLX@@@acls[#1](#2)#3{%
  \GLX@do@glosstex{#3}{#1}{#2}{a}\GLX@do@acl{#3}{#2}}
\newcommand{\GLX@@@aclr}{}
\def\GLX@@@aclr(#1)[#2]#3{%
  \GLX@do@glosstex{#3}{#2}{#1}{a}\GLX@do@acl{#3}{#1}}
\newcommand{\GLX@aclS}{%
  \@ifnextchar[{\GLX@@aclSs}{%
    \@ifnextchar({\GLX@@aclSr}
    {\GLX@@@aclSs[\GLX@acdef@page](\GLX@acdef@list)}}}
\newcommand{\GLX@@aclSs}{}
\def\GLX@@aclSs[#1]{%
  \@ifnextchar({\GLX@@@aclSs[#1]}{\GLX@@@aclSs[#1](\GLX@acdef@list)}}
\newcommand{\GLX@@aclSr}{}
\def\GLX@@aclSr(#1){%
  \@ifnextchar[{\GLX@@@aclSr(#1)}{\GLX@@@aclSs[\GLX@acdef@page](#1)}}
\newcommand{\GLX@@@aclSs}{}
\def\GLX@@@aclSs[#1](#2)#3{%
  \GLX@do@glosstex{#3}{#1}{#2}{n}\GLX@do@acl{#3}{#2}}
\newcommand{\GLX@@@aclSr}{}
\def\GLX@@@aclSr(#1)[#2]#3{%
  \GLX@do@glosstex{#3}{#2}{#1}{n}\GLX@do@acl{#3}{#1}}
%    \end{macrocode}
% \end{macro}
% \end{macro}
%
% \begin{macro}{\acf}
% \begin{macro}{\acf*}
%    \begin{macrocode}
\DeclareRobustCommand\acf{%
  \@ifstar{\GLX@acfS}{\GLX@acf}}
\newcommand{\GLX@acf}{%
  \@ifnextchar[{\GLX@@acfs}{%
    \@ifnextchar({\GLX@@acfr}
    {\GLX@@@acfs[\GLX@acdef@page](\GLX@acdef@list)}}}
\newcommand{\GLX@@acfs}{}
\def\GLX@@acfs[#1]{%
  \@ifnextchar({\GLX@@@acfs[#1]}{\GLX@@@acfs[#1](\GLX@acdef@list)}}
\newcommand{\GLX@@acfr}{}
\def\GLX@@acfr(#1){%
  \@ifnextchar[{\GLX@@@acfr(#1)}{\GLX@@@acfs[\GLX@acdef@page](#1)}}
\newcommand{\GLX@@@acfs}{}
\def\GLX@@@acfs[#1](#2)#3{%
  \GLX@do@glosstex{#3}{#1}{#2}{a}\GLX@do@acf{#3}{#2}}
\newcommand{\GLX@@@acfr}{}
\def\GLX@@@acfr(#1)[#2]#3{%
  \GLX@do@glosstex{#3}{#2}{#1}{a}\GLX@do@acf{#3}{#1}}
\newcommand{\GLX@acfS}{%
  \@ifnextchar[{\GLX@@acfSs}{%
    \@ifnextchar({\GLX@@acfSr}
    {\GLX@@@acfSs[\GLX@acdef@page](\GLX@acdef@list)}}}
\newcommand{\GLX@@acfSs}{}
\def\GLX@@acfSs[#1]{%
  \@ifnextchar({\GLX@@@acfSs[#1]}{\GLX@@@acfSs[#1](\GLX@acdef@list)}}
\newcommand{\GLX@@acfSr}{}
\def\GLX@@acfSr(#1){%
  \@ifnextchar[{\GLX@@@acfSr(#1)}{\GLX@@@acfSs[\GLX@acdef@page](#1)}}
\newcommand{\GLX@@@acfSs}{}
\def\GLX@@@acfSs[#1](#2)#3{%
  \GLX@do@glosstex{#3}{#1}{#2}{n}\GLX@do@acf{#3}{#2}}
\newcommand{\GLX@@@acfSr}{}
\def\GLX@@@acfSr(#1)[#2]#3{%
  \GLX@do@glosstex{#3}{#2}{#1}{n}\GLX@do@acf{#3}{#1}}
%    \end{macrocode}
% \end{macro}
% \end{macro}
%
% \begin{macro}{\GLX@entry}
% This is the control-sequence written into the |.aux|-file which is
% read by \GlossTeX{}. This
% macro is defined so that it takes 5 arguments and expands to
% |\relax| when the |.aux|-file is reread by \TeX{}.
%    \begin{macrocode}
\newcommand{\GLX@entry}[5]{\relax}
%    \end{macrocode}
% \end{macro}
%
% \begin{macro}{\GLX@do@glosstex}
% This macro writes lines of the format
% \begin{definition}
%   |\GLX@entry{|\meta{label}|}{|\meta{pageref-mode}|}{|\meta{list}|}{|\meta{list-mode}|}{|\meta{page}|}|
% \end{definition}
% into the |.aux|-file.
%    \begin{macrocode}
\newcommand{\GLX@do@glosstex}[4]{%
  \protected@write\@auxout{}%
  {\string\GLX@entry{#1}{#2}{#3}{#4}{\thepage}}}
%    \end{macrocode}
% \end{macro}
%
% The following commands create macros so that |\GLX@term@|\meta{label} contains
% the short-form of \meta{label} |#1|, |\GLX@term@@|\meta{label} contains the \meta{long-form}
% of \meta{label} |#1| and |\GLX@terml@|\meta{label} expands to |x| if the \meta{long-form} of
% \meta{label} |#1| has already been used. This is heavily borrowed from
% \package{acronmy}.
%
% \begin{macro}{\GLX@do@acs}
% Output the short-form \meta{item} of \meta{label} |#1| in \meta{list} |#2|.
%    \begin{macrocode}
\newcommand{\GLX@do@acs}[2]{%
  \expandafter\ifx\csname GLX@term@#1\endcsname\relax%
    \textbf{#1!}%
    \PackageWarning{glosstex}{Term `#1' is not defined}%
    \expandafter\gdef\csname GLX@term@#1\endcsname{\textbf{#1!}}%
  \else%
    \GLX@output@short{#1}{#2}{\csname GLX@term@#1\endcsname}%
  \fi}
%    \end{macrocode}
% \end{macro}
%
% \begin{macro}{\GLX@do@acl}
% Output the \meta{long-form} of \meta{label} |#1|.
%    \begin{macrocode}
\newcommand{\GLX@do@acl}[2]{%
  \expandafter\ifx\csname GLX@term@@#1\endcsname\relax%
    \textbf{#1!}%
    \PackageWarning{glosstex}{Term `#1' has no long form}%
    \expandafter\gdef\csname GLX@term@@#1\endcsname{\textbf{#1!}}%
  \else%
    \GLX@output@long{#1}{#2}{\csname GLX@term@@#1\endcsname}%
  \fi}
%    \end{macrocode}
% \end{macro}
%
% \begin{macro}{\GLX@do@acf}
% Output \meta{short} |(|\meta{long}|)| of \meta{label} |#1| by calling
% |\GLX@do@acs| and |GLX@do@acl|.
%    \begin{macrocode}
\newcommand{\GLX@do@acf}[2]{%
  \GLX@do@acs{#1}{#2}%
  \expandafter\ifx\csname GLX@term@@#1\endcsname\relax%
  \else%
    \nolinebreak[3]\space(\GLX@do@acl{#1}{#2})%
  \fi}%
%    \end{macrocode}
% \end{macro}
%
% \begin{macro}{\GLX@do@ac}
% Either call |\GLX@do@acf| or |\GLX@do@acs|, depending on
% |\GLX@terml@#1| $==$ |x|.
%    \begin{macrocode}
\newcommand{\GLX@do@ac}[2]{%
  \expandafter\ifx\csname GLX@terml@#1\endcsname\relax%
    \GLX@do@acf{#1}{#2}%
    \expandafter\gdef\csname GLX@terml@#1\endcsname{x}%
  \else%
    \GLX@do@acs{#1}{#2}%
  \fi}
%    \end{macrocode}
% \end{macro}
%
% \begin{macro}{\GLX@newterm}
% This macro defines |\GLX@term@#1| et.\ al so that they expand to
% their appropriate meaning.
%    \begin{macrocode}
\newcommand{\GLX@newterm}[3]{%
  \expandafter\gdef\csname GLX@term@#1\endcsname{#2}
  \ifx#3\empty%
  \else\expandafter\gdef\csname GLX@term@@#1\endcsname{#3}\fi}
%    \end{macrocode}
% \end{macro}
%
% \begin{environment}{theglosstex}
% This environment is used for typesetting the lists generated by
% \GlossTeX{}. These are the meanings of the parameters:
%
% \begin{tabular}{ll}
% |#1| & This \meta{list}\\
% |#2| & This \meta{pageref-mode}\\
% |##1| & \meta{label}\\
% |##2| & \meta{item}\\
% |##3| & \meta{long-form}\\
% |##4| & \meta{text}\\
% |##5| & \meta{list}\\
% |##6| & \meta{list-mode}\\
% |##7| & \meta{page-stuff}\\
% \end{tabular}
%
% The whole list is encapsulated into the environment defined by the
% macros |\GLX@benv@|\meta{list} and |\GLX@eenv@|\meta{list}. These
% contain |\begin{description}| and |\end{description}| for
% \meta{list} $==$ |(glo)| as default. Each |\GlossTeXEntry| is only
% processed if it's \meta{list} is identical to the environment's
% \meta{list} and it's \meta{list-mode} $==$
% |[a]|. |\GLX@item@|\meta{list} is called for each of these entries.
%    \begin{macrocode}
\newenvironment{theglosstex}[2]{%
  \gdef\GLX@theglosstex@list{#1}%
  \gdef\GLX@theglosstex@pagerefmode{#2}%
  \csname GLX@benv@#1\endcsname%
%    \end{macrocode}
%
% \begin{macro}{\glxref}
% \begin{macro}{\glxrefS}
% This one produces an entry in the current \meta{list} (stored in
% |\GLX@theglosstex@list|). 
%    \begin{macrocode}
  \DeclareRobustCommand{\glxref}{%
    \@ifstar{\GLX@glxrefS}{\GLX@glxref}}
  \newcommand{\GLX@glxref}[1]{%
    \GLX@do@glosstex{##1}{n}{\GLX@theglosstex@list}{a}%
    \GLX@do@acs{##1}{\GLX@theglosstex@list}}
  \newcommand{\GLX@glxrefS}[1]{%
    \GLX@do@glosstex{##1}{n}{\GLX@theglosstex@list}{a}}
%    \end{macrocode}
% \end{macro}
% \end{macro}
%
%    \begin{macrocode}
  \newcommand{\GlossTeXEntry}[7]{%
    \GLX@newterm{##1}{##2}{##3}%
    \write\@auxout{\string\GLX@newterm{##1}{##2}{##3}}%
    \ifthenelse{\equal{#1}{##5}}{%
      \ifthenelse{\equal{a}{##6}}{%
        \csname GLX@item@#1\endcsname%
          {##1}{##2}{##3}{##4}{##5}{##6}{##7}}{}%
      }{}}}{%
  \csname GLX@eenv@\GLX@theglosstex@list\endcsname}
%    \end{macrocode}
% \end{environment}
%
% \begin{macro}{\printglosstex}
% The optional arguments are the \meta{list} to be typeset and the
% \meta{pageref-mode}. First, 
% |\GLX@heading@|\meta{list} is called (defaults to something like
% |\section{\glossaryname}|) to produce a proper heading. Then, if
% present, the |.glx|-file is included into a |theglosstex| environment.
%    \begin{macrocode}
\DeclareRobustCommand\printglosstex{\GLX@printglosstex}
\newcommand{\GLX@printglosstex}{%
  \@ifnextchar[{\GLX@@printglosstexs}{%
    \@ifnextchar({\GLX@@printglosstexr}
    {\GLX@@@printglosstexs[p](glo)}}}
\newcommand{\GLX@@printglosstexs}{}
\def\GLX@@printglosstexs[#1]{%
  \@ifnextchar({\GLX@@@printglosstexs[#1]}
  {\GLX@@@printglosstexs[#1](glo)}}
\newcommand{\GLX@@printglosstexr}{}
\def\GLX@@printglosstexr(#1){%
  \@ifnextchar[{\GLX@@@printglosstexr(#1)}
  {\GLX@@@printglosstexs[p](#1)}}
\newcommand{\GLX@@@printglosstexs}{}
\def\GLX@@@printglosstexs[#1](#2){\GLX@do@printglosstex{#2}{#1}}
\newcommand{\GLX@@@printglosstexr}{}
\def\GLX@@@printglosstexr(#1)[#2]{\GLX@do@printglosstex{#1}{#2}}
\newcommand{\GLX@do@printglosstex}[2]{%
  \expandafter\ifx\csname GLX@heading@#1\endcsname\relax%
    \textbf{#1!}%
    \PackageWarning{glosstex}{Heading for list `#1' is not defined}%
    \expandafter\gdef\csname GLX@heading@#1\endcsname{\textbf{#1!}}%
  \else%
    \csname GLX@heading@#1\endcsname%
    \IfFileExists{\jobname.glx}{%
      \begin{theglosstex}{#1}{#2}%
        \input{\jobname.glx}%
      \end{theglosstex}}%
    {\typeout{No file \jobname.glx}}%
  \fi}
%    \end{macrocode}
% \end{macro}
%
% \begin{macro}{\glxgldefault}
% \begin{macro}{\glxacdefault}
% The commands set the default \meta{list} and \meta{pageref-mode} for
% the commands starting with |\gl| and |\ac|, respectively.
%    \begin{macrocode}
\newcommand{\glxgldefault}[2]{%
  \gdef\GLX@gldef@list{#1}%
  \gdef\GLX@gldef@page{#2}}

\newcommand{\glxacdefault}[2]{%
  \gdef\GLX@acdef@list{#1}%
  \gdef\GLX@acdef@page{#2}}
%    \end{macrocode}
% \end{macro}
% \end{macro}
%
% \begin{macro}{\glxheading}
% This macro sets the headings to be used typesetting a \meta{list}.
%    \begin{macrocode}
\newcommand{\glxheading}[2][glo]{%
  \expandafter\gdef\csname GLX@heading@#1\endcsname{#2}}
%    \end{macrocode}
% \end{macro}
%
% Now include the mandatory standard configuration file |glosstex.std|
% and an optional |.cfg|-file.
%    \begin{macrocode}
\InputIfFileExists{glosstex.std}{%
  \typeout{Using the standard configuration file glosstex.std}}{%
  \PackageWarning{glosstex}{%
    File glosstex.std not found. Expect many errors}}
\InputIfFileExists{glosstex.cfg}{%
  \typeout{Using the configuration file glosstex.cfg}}{}
%</package>
%    \end{macrocode}
%
% \subsection{The \MakeIndex{} style-file \file{glosstex.mst}}
%
% It is very important to use an appropriate \gls{mst-file} for \MakeIndex\
% to be able to read and write files in the correct format. \GlossTeX\
% comes with |glosstex.mst| which has to be used as style-file for
% \MakeIndex{}. The file |glosstex.mst| should not be modified since
% it is tightly coupled to the features of \MakeIndex{}. Note the use
% of ``|~|'' as |quote| to enable german word-ordering using the |-g|
% option to \MakeIndex{}. So if you want a ``|~|'' in your |.gdf|-file,
% you have to type ``|~~|''
%
%    \begin{macrocode}
%<*mst>
keyword        "\\GlossTeXEntry"
quote          '~'                     
delim_0        ""                      
group_skip     "\n"
indent_length  0
indent_space   ""
item_0         "\n\\GlossTeXEntry"
postamble      "\n"
preamble       "\n"
%</mst>
%    \end{macrocode}
%
% \Finale
\endinput
